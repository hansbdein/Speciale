\chapter{Output}
\label{chap:Results}
\begin{figure}[ht]
    %\centering
    \includegraphics[width=5.1in]{figures/abundancelight.pdf}
    \caption{Time evolution of light nuclei abundance during BNN, with mass fraction technically being the close approximation $X_i=Y_i A_i$ }
    \label{fig:lightXevo}
\end{figure}
Running the code creates a complete overview of abundance evolution as shown on figure \ref{fig:lightXevo}. As expected neutrons protons and deuterium remain in equilibrium for the first few seconds. As the temperature decreases, deuterium abundances slowly increase quickly followed by tritium and both helium isotopes. %$2{}^2\text{H}\rightarrow {}^4\text{He}$ has a very small cross-section With ${}^4$H initially being produced primarily by ${}^3\text{H}+p\text{H}\rightarrow {}^4\text{He}$ and later by ${}^3\text{H}+{}^2\text{H}\rightarrow {}^4\text{He}+n$.
This continues until around 230 seconds, at which point the rate of ${}^4$He creation is finally great enough to have a significant impact on neutron abundance, which until then had remained almost unchanged since the $p\leftrightarrow n$ rates fell out of equilibrium. This leads to a rapid drop in neutron abundance creating a bottleneck on the production of deuterium and tritium. Without neutron capture to create more, the existing deuterium and tritium is converted to ${}^4$He via ${}^3\text{H}+{}^2\text{H}\rightarrow {}^4\text{He}+n$. Without these light nuclei lithium abundances also drop, as reactions such as ${}^4\text{He}+{}^3\text{H}\rightarrow {}^7\text{Li}$ become outmatched by the proton captures ${}^7\text{Li}+\text{p}\rightarrow 2{}^4\text{He}$ and ${}^6\text{Li}+\text{p}\rightarrow {}^7\text{Be}$. Conversely, Beryllium 7 is primarily destroyed via neutron capture ${}^7\text{Be}+\text{n}\rightarrow {}^7\text{Li}+\text{p}$, and therefore sees a rapid increase in abundance immediately after the drop in neutrons.
\begin{figure}[ht]
    %\centering
    \includegraphics[width=5.1in]{figures/smallnet5minutes.pdf}
    \caption{Reaction rates 5 minutes after Big Bang at \num{7.6e8} K. Only including rates within 10 orders of magnitude of strongest. Each rate is represented by arrows from each reactant to each product}
    \label{fig:5minutenet}
\end{figure}

The relations of these reactions are illustrated in figure \ref{fig:5minutenet}. This snapshot is taken immediately after the aforementioned drop in neutron abundance with $Y_n=0.2\%$. Despite this $n+p\rightarrow d$ is still the strongest reaction rate, followed by rates creating ${}^3$He and ${}^3$H, simply due to the high abundance of these light nuclei. Though no particular reaction is exceptionally strong, a disproportionate amount of reactions produce rather than consume ${}^4$He. This should come as no surprise as ${}^4$He is the most tightly bound of all these nuclei. The only rates consuming helium are those producing heavier elements, ${}^4\text{He}+{}^3\text{H}\rightarrow {}^7\text{Li}$, ${}^4\text{He}+{}^3\text{He}\rightarrow {}^7\text{Be}$, and ${}^4\text{He}+{}^2\text{H}\rightarrow {}^6\text{Li}$. Yet these barely affect total helium abundance, and to an extent help create more ${}^4$He through subsequent reactions such as ${}^7\text{Li}+{}^2\text{H}\rightarrow 2 {}^4\text{He}+\text{n}$. 


\begin{figure}[ht]
    %\centering
    \includegraphics[width=5.1in]{figures/abundanceheavy.pdf}
    \caption{Time evolution of heavy nuclei abundance during BNN, with mass fraction technically being the close approximation $X_i=Y_i A_i$}
    \label{fig:heavyXevo}
\end{figure}

The most striking effect of the high ${}^4$He binding energy is the lack of stable nuclei with $A=8$, due 2 alpha particles being energetically favorable. Many nearby nuclei are also unstable creating a gap in the full reaction network, apparent on \ref{fig:bignet}. The next nucleus with greater binding energy per nucleon is ${}^{12}$C, which after BBN is also the most abundant of the heavy nuclei, as seen on figure \ref{fig:heavyXevo}. Bridging the gap between ${}^4$He and ${}^{12}$C is usually accomplished via the triple-alpha process $3{}^4\text{He}\rightarrow {}^{12}\text{C}$, which occurs in stars at temperatures above $10^8$K. The temperature during BBN is even higher than this, but compared to stellar interiors the density is much lower. At the previously mentioned 5 minute mark, the baryon density of the universe is only 10 grams per cubic meter, which is more than ten orders of magnitude lower than that of a helium burning stellar core. This completely stops the triple-alpha process, leaving the inefficient process of alpha capture on ${}^7$Li and later ${}^7$Be as the only options for creating heavier nuclei. This firstly creates ${}^{11}$B, which is through proton capture is responsible for the majority of ${}^{12}$C production. Unfortunately the resulting exited state ${}^{12}$C* predominately decays into three alpha particles, with the branching ratio of internal translation into the ground state being only \num{1.5e-4}. The same is true for neutron capture on ${}^{11}$C, though it also decays via proton emission  ${}^{12}\text{C}^\ast \rightarrow {}^{11}\text{B}+\text{p}$ and of course freely decays ${}^{11}\text{C}\rightarrow {}^{11}\text{B}$. From figure \ref{fig:heavyXevo} we also see an initial bump in ${}^9$Be abundance caused by the uniquely high cross-section of the reaction ${}^7\text{Li}+{}^3\text{H}\rightarrow {}^9\text{Be}+n$, but as tritium abundance drops most ${}^9$Be is destroyed by photofission $\gamma+{}^9\text{Be}\rightarrow 2{}^4\text{He}+n$.


\section{Precision}

Throughout the code there are several places where numerical accuracy must be balanced by the need for swift computation. As a baseline I aim for a relative precision on final abundances of $10^{-5}$ due to numerical uncertainty. This is much lower than the error introduced by the experimental determination of reaction rates, as well as that of astronomical observations of primordial abundances. 

\subsection{Tolerances}
For the integration itself I use the solve\_ivp method from scipy.integrate\cite{SciPy}. Here ensuring numerical accuracy is a simple matter of specifying the desired relative and absolute tolerances as optional parameters, which I have set as  rtol=1e-6 and atol=1e-80. The absolute tolerance might seem exorbitantly low, but this helps improve stability since some heavy nuclei can cause detrimental transients if not tracked accurately at even minute abundances. However, for the tests this section I use rtol=1e-8, to ensure any inaccuracy is caused be whatever parameter is being tested.


\subsection{Electron energy}
Most calculations for the background have nice analytical expressions that can be computed with arbitrary precision. The electron energy density is the main exception, as it is given by an infinite sum \eqref{eq:rhoelectron} of which a finite number of terms must be calculated. Electrons annihilate very early, so any error in the energy density will only affect the $p\leftrightharpoons n$ rate. This will lead to a small change in neutron abundance, which results in a similar (within one order of magnitude) relative change in the abundance of all subsequent nuclei. 
Testing shows that small changes to the energy density, cause a change in neutron abundance of roughly the same relative magnitude if not slightly lower. So to achieve the requested accuracy in abundance, I will also aim for a $10^{-5}$ relative error in total energy density. 
\begin{figure}[ht]
    \includegraphics[width=5.1in]{figures/Besselaccuracy.pdf}
    \caption{Normalized absolute deviation of electron energy density from the reference energy as a function of included terms in \eqref{eq:rhoelectron}. The reference energy density has been determined by computing the first 10000 terms}
    \label{fig:Besselaccuracy}
\end{figure}

Figure \ref{fig:Besselaccuracy} shows the deviation from the reference energy at select temperatures before and during e$^-$ e$^+$ annihilation. For low temperatures the sum converges quickly, which should come as no surprise as the series in \eqref{eq:electronseries}, converges more rapidly as $e^{-(E\pm\mu)/T}$ approaches 0. At high temperatures the sum converges at a much slower rate, but it still achieves sufficient accuracy with only 20 terms, which is the number I will use moving forward.

\subsection{Interpolation}
\label{sec:interpolation}
For the interpolation on background variables an appropriate number of points must be selected. These points are spaced uniformly in log(t). Tests show that $10^4$ points is sufficient to achieve the required accuracy in final abundances. However, a low number of points cause the small discontinuities in the derivatives of temperature and scale factor. Due to the stiffness of the system this leads to instabilities, which have a major impact on the runtime. At the required resolution of $10^4$ the integration routine has to make twice as many call to the RHS, compared to smoother integration. At $2\times 10^5$ points the interpolations is smooth enough as to not cause issues for integration.


\subsection{Network timings}
\label{sec:networktiming}
As  mentioned in section \ref{sec:pna}, we cannot calculate the abundance of all nuclei from the beginning, and have to add them gradually as the y become relevant. To find this point we can plot the deviation in final abundance as a function of the time at which we switch between networks.
\subsubsection{Full network}
We start by looking at the full network shown on figure \ref{fig:bignettime}, which has three distinct points. As long as the heavy nuclei are added before 2 minutes corresponding to $T>10^9$K they have time to reach the same abundance as they would if they were added earlier. Adding them later will delay the initial production, which will slowly lower the resulting final abundance. They will however still reach approximately the same final abundance until the late stages of BBN at around 7 minutes. Here the temperature will be to low for their formation, resulting in an increase in Lithium and Beryllium, as these are no longer consumed to produce heavy nuclei. The reaction involving ${}^{11}$C however come into effect much later than most other reactions and remain relevant even an hour after BBN. If ${}^{11}$C is added after this point, it will never achieve a significant final abundance, leading to a drop in its indirect daughter nuclei ${}^{9}$Be and ${}^{6}$Li. 
\begin{figure}[ht]
    \includegraphics[width=5.1in]{figures/Bignettime.pdf}
    \caption{Relative deviation on final abundances based on the time from which heavy nuclei are included in the reaction network. To display both positive and negative changes, relative deviations $ <10^{-8}$ are excluded. The bottom shows negative deviation on a linear scale}
    \label{fig:bignettime}
\end{figure}

The result of this test shows that for standard BBN heavy nuclei don't actually need to be included at all, if one is only interested in abundances of light nuclei. For completeness, I will choose to include these nuclei anyway starting at $T>10^9$K. 

\subsubsection{Light network}

From figure \ref{fig:midnettime} we see that delaying the calculation of nuclear abundances, initially cause an increase abundance of protons and the proton dominated nuclei ${}^3$He and ${}^8$B, with all other abundances dropping. This is due to early nuclear reactions trapping neutrons in the cores light nuclei such as ${}^4$He and ${}^2$H, preventing their free decay into protons. However, since the vast majority of neutrons are still unbound, this effect will be insignificant until large light nuclei abundances are attained. Therefore, we actually don't need to calculate early abundances to get precise estimates of the final abundance. 

\begin{figure}[ht]
    \includegraphics[width=5.1in]{figures/midnettime.pdf}
    \caption{Relative deviation on final abundances based on the time from which light nuclei are added to the reaction network. Plot is linear in the region $\pm 10^{-7}$ and logarithmic outside}
    \label{fig:midnettime}
\end{figure}
Not having to solve the reaction network at high temperatures makes the system much less stiff, potentially allowing future BBN codes to use less complex integrations routines than those used by this and older codes. Unfortunately starting abundance calculations presents a major issue due to the initial conditions. When starting light abundance calculation at early times $t<10$s, we can determine the initial abundances by setting the RHS to 0 as described in section \ref{sec:structure}. But at later times reactions destroying ${}^4$He drop off, with ${}^4$He abundances instead being kept in check by the lack of deuterium. As the system is no longer in exact NSE, any method for determining initial condition which assume a steady state solution will fail. Given enough time all neutrons will be converted to ${}^4$He, and the estimated initial conditions reflect this by vastly overestimating ${}^4$He abundances. To get around this and produce figure \ref{fig:midnettime}, I instead had to set all initial abundances to 0 when initiating the light network after 3 seconds. This workaround eliminates the problem, but creates another in the form of transients create as abundances go from 0 to their appropriate value. Since the system of equations is less stiff at these late times the transients won't break the integration. But dealing with these transients takes just as long as the time required for just tracking abundances at early times. 


\subsubsection{$p\leftrightharpoons n$ network}
Unlike the heavier nuclei the at which we start calculating neutrons and protons abundances have a very abrupt and significant impact on final abundances. As long as they are still in thermal equilibrium when we begin calculations, the final abundance will be the same. Conversely, if we start after they begin to fall out of thermal equilibrium, all final abundances will be radically different. From figure \ref{fig:npnettime} it's clear that this occurs at around 100 ms or \num{3e10}K, and as long as we initiate $p\leftrightharpoons n$ before this we get precise results. This also justifies the use of 27e9 K as the initial temperature, which is the point used by among others Wagoner, Kawano, and AlterBBN.
\begin{figure}[ht]
    \includegraphics[width=5.1in]{figures/npnettime.pdf}
    \caption{Relative deviation of final abundances based on the initial time of BBN calculations. Plot is linear in the region $\pm 10^{-7}$ and logarithmic outside}
    \label{fig:npnettime}
\end{figure}



\subsection{Neutrino decoupling}

The main source of inaccuracy in this code is the assumption of complete neutrino decoupling. This assumption is reasonable as the only net transfer of energy between different particles is annihilation of electrons and positrons, which begins at around 1 MeV as shown on figure \ref{fig:rhoegammaT}. This is long after the neutrinos have decoupled, and therefore the energy is transferred to the photons, which are heated relative to the neutrinos by the well known factor $\sqrt[3]{11/4}$. There is however a small transfer of energy before the neutrinos completely decouple from the other particles. The result of this is a small increase in neutrino temperature relative to photon temperature of less than 1\% \cite{Hannestad:1995rs}. As both neutrinos and photons are ultrarelativistic, this change is inconsequential to the time evolution of the early universe, and should intuitively not have a significant impact on final abundances. This is however not the case, as is apparent from figure \ref{fig:initime}. The reason for this is that the higher photon temperature caused by instantaneous decoupling, effectively delays the onset of nucleosynthesis, since the baryons take longer to cool sufficiently. This has two main effects, the first being slightly lower baryon density at any given temperature, and the second that the universe will remain in any given temperature range for longer. Additionally, neutrino decoupling also affects the $p\leftrightharpoons n$ reactions directly, as these rely on both exact temperature and non-thermal energy distribution of the neutrinos. When assuming complete neutrino decoupling, these effects lowers the abundance of ${}^4$He by 0.05\%, ${}^3$He by 0.1\%, and deuterium by 0.5\%, and increase ${}^7$Be by 0.5\%, (PRIMAT table V\cite{PRIMAT}). 


Since my implementation uses estimated $p\leftrightharpoons n$ rates which do take into account incomplete neutrino decoupling, my results are primarily affected by the delaying of nucleosynthesis, caused by the inaccurate $T_\nu/T\gamma$ ratio. To reduce the deviation in $T_\nu/T\gamma$ ratio, we can decouple the neutrinos as close as possible to the point at which they actually decouple. This still assumes an instant rather than gradual decoupling, but nonetheless improves predicted abundances. The point at which neutrinos decouple obviously coincide with the freezeout of $p\leftrightharpoons n$ reactions, and as shown on \ref{fig:npnettime}, we can safely begin both background and network integration at 27 GK corresponding to 2.3 MeV. 
\begin{table}[ht]
    \begin{tabular}{l|llllll}
        & $Y_p \times 10$ & \hspace{-0.34em}$^{2}$H$ \times 10^{5}$ & \hspace{-0.34em}$^{3}$He$ \times 10^{5}$ & \hspace{-0.34em}$^{7}$Li $ \times 10^{10}$& \hspace{-0.34em}$^{6}$Li $ \times 10^{15}$& \hspace{-0.34em}$^{7}$Be $ \times 10^{10}$\\ \hline
    100 MeV & 2.470            & 2.532 & 1.040 & 4.712 & 7.503 & 4.411     \\ \hline
    2.3 MeV  & 2.472            & 2.536 & 1.040 & 4.706 & 7.523 & 4.404   \\ \hline
    Deviation & 0.07\%           & 0.15\% & 0.04\% & -0.15\% & 0.25\% & -0.18\%      
    \end{tabular}
    \caption{Comparison of final abundances for instant decoupling at 2.3 and 10 MeV, with all abundances except ${}^4$He ($Y_p$) being normalized to H abundance. ${}^3$He and ${}^7$Li including contribution from eventual decay of ${}^3$H and ${}^7$Be}
    \label{tab:earlylatedecoup}
\end{table}

From table \ref{tab:earlylatedecoup} we clearly see the impact of neutrino decoupling. Compared to the expected changes determined by PRIMAT\cite{PRIMAT}, we see an increased impact on ${}^4$He and a reduced impact on ${}^7$Be and ${}^3$He. This can mostly be explained by the $p\leftrightharpoons n$ reactions not being directly impacted in my implementation. The direct effect on $p\leftrightharpoons n$ increases the neutron abundance subsequently the abundance of all other nuclei, counteracting the effects on everything but ${}^7$Be, as well as ${}^3$He since its abundance actually increases as neutron abundance drops.%The direct effect only changes the neutron proton ratio, which partially cancels the changes introduced by the delay of nucleosynthesis, particularly in ${}^4$He abundance. Therefore, the theoretical systematic error in ${}^4$He is slightly higher, which we will explore further in section \ref{sec:Altercompare}.
\begin{figure}[ht]
    \includegraphics[width=5.1in]{figures/initime.pdf}
    \caption{Relative deviation on final abundances based on the initial time for background calculations. By construction the initial time coincides with the time of neutrino decoupling}
    \label{fig:initime}
\end{figure}

Though the errors associated neutrino decoupling are much greater than the achieved numerical error of less than $10^{-5}$, they are still acceptable, since they are completely inconsequential next to the uncertainty associated with reactions rates, as will become apparent in the following section.


\section{Comparison with AlterBBN}
\label{sec:Altercompare}
To compare the results of this code to that of other contemporary BBN codes, I have chosen AlterBBN. The main reason for this is that my code is written in Python with much of the heavy computation being done directly in C. This makes it easier to interface with the C based AlterBBN than PArthENoPE or PRIMAT, which as previously mentioned are written in FORTRAN and Mathematica. 
\begin{table}[ht]
    \begin{tabular}{l|llllll}
        & $Y_p \times 10$ & \hspace{-0.34em}$^{2}$H$ \times 10^{5}$ & \hspace{-0.34em}$^{3}$He$ \times 10^{5}$ & \hspace{-0.34em}$^{7}$Li $ \times 10^{10}$& \hspace{-0.34em}$^{6}$Li $ \times 10^{14}$& \hspace{-0.34em}$^{7}$Be $ \times 10^{10}$\\ \hline
    This work & 2.472            & 2.536 & 1.040 & 4.706 & 0.752 & 4.404   \\ \hline
    AlterBBN & 2.474            & 2.467 & 1.034 & 5.363 & 1.087 & 5.075   \\ %\hline
    $\quad +/-$ & 0.003           & 0.038 & 0.016 & 0.352 & 1.085 & 0.343      
    \end{tabular}
    \caption{Abundances from this work and AlterBBN, with all abundances except ${}^4$He ($Y_p$) being normalized to the H abundance. ${}^3$He and ${}^7$Li including contribution from eventual decay of ${}^3$H and ${}^7$Be. Quoted uncertainties are those determined by AlterBBN}
    \label{tab:shortAlterabun}
\end{table}


Comparing the final abundances predicted by this code and AlterBBN we see major discrepancies. Most of these are within the uncertainties calculated by AlterBBN based on the uncertainty of the reaction rates. ${}^7$Be and subsequently ${}^7$Li deviate by more than twice what is predicted, and the same is true for deuterium. To discover the cause of this we can do a 1 to 1 comparison of the time evolution of the universe according to this work and AlterBBN. Here we use the exact same initial conditions of $T=27\times10^9$K, $\eta=6.1$e-$10$ and $\tau_n=880.2$s, remembering to correct for the inaccurate initial time in AlterBBN as explained in section \ref{sec:t_ini}.

\subsection{Background}
To rule out cosmological differences, we can compare the predicted background parameters from AlterBBN and this code, displayed on figure \ref{fig:comparetemp}. Unsurprisingly after the common initial temperature of 27GK, AlterBBN sees a relative increase in neutrino temperature due to incomplete neutrino decoupling, with photon temperature initially decreasing slower due to sharing temperature with the electrons. Later there is a small additional drop in photon temperature coinciding with the onset of primary nucleosynthesis. This can be explained by AlterBBN taking into account the effect of nucleosynthesis on photon temperature, due to the energy release of nucleosynthesis. However, the observed effect is unphysical and must be due to numerical errors, since the energy release is nowhere near the amount required for an impact on photon temperature of this size, and additionally it should serve to increase, rather than decrease the photon temperature. Lastly, there is a relative decrease of the AlterBBN neutrino temperature at late times. This is due to AlterBBN taking into account both baryon and dark matter energy density, which I have omitted since it clearly doesn't have a significant impact during the era of BBN.
\begin{figure}[ht]
    \includegraphics[width=5.1in]{figures/comparetemp.pdf}
    \caption{Relative deviation of temperatures as calculated by this code and AlterBBN. Positive values signify higher AlterBBN temperatures compared to this code}
    \label{fig:comparetemp}
\end{figure}

To see the impact of these background differences we can integrate the reaction network using the background parameters from AlterBBN. At its highest precision setting AlterBBN takes 64421 discreet time steps, which based the test in section \ref{sec:interpolation} is sufficient to achieve high accuracy with the existing interpolation routine. AlterBBN does however not directly track the scale factor, so we must instead track the baryon density using the neutrino temperature. Since the tabulated $p\leftrightharpoons n$ rates used in this network only depend on temperature, we don't need to know the exact value of $n_b$ before the other reactions become relevant at around 100 seconds, (figure \ref{fig:midnettime}). At this time neutrinos have actually decoupled completely, and therefore the relation $T_\nu \propto a^{-1}$ holds exactly. This allows the baryon number density to be determined by $n_b= C \cdot T_\nu^3$ with the constant $C$ being set by requiring $\eta=6.1$e-$10$ at late times.
\begin{table}[ht]
    \begin{tabular}{l|llllll}
               & $Y_p \times 10$ & \hspace{-0.34em}$^{2}$H$ \times 10^{5}$ & \hspace{-0.34em}$^{3}$He$ \times 10^{5}$ & \hspace{-0.34em}$^{7}$Li $ \times 10^{10}$& \hspace{-0.34em}$^{6}$Li $ \times 10^{14}$& \hspace{-0.34em}$^{7}$Be $ \times 10^{10}$\\ \hline
    100 MeV & 2.470            & 2.532 & 1.040 & 4.712 & 0.750 & 4.411     \\ \hline
    2.3 MeV  & 2.472            & 2.536 & 1.040 & 4.706 & 0.752 & 4.404    \\ \hline
    \begin{tabular}[c]{@{}l@{}}This work using \\ AlterBBN $T_\gamma$ \& $T_\nu$ \end{tabular} & 2.473            & 2.540 & 1.041 & 4.699 & 0.754 & 4.398   \\ \hline
    AlterBBN & 2.474            & 2.467 & 1.034 & 5.363 & 1.087 & 5.075   \\ %\hline
    $\quad +/-$ & 0.003           & 0.038 & 0.016 & 0.352 & 1.085 & 0.343     
    \end{tabular}
    \caption{Comparison of final abundances for various background parameters. Showing results of this network using early and late instant neutrino decoupling, as well as the AlterBBN background which accounts for incomplete decoupling. The last row shows result from the AlterBBN background using their own abundance calculations}
    \label{tab:longAlterabun}
\end{table}

The result of using the background parameters of AlterBBN are displayed on Table \ref{tab:longAlterabun}. Though somewhat obscured by the lack of significant digits, relative change caused by using the AlterBBN background is equal to 90\% of the change caused by changing the timing of instant neutrino decoupling. This is due to the increase in $T_\nu/T_\gamma$ predicted by full incomplete decoupling, being 1.9 times greater than what is achieved by simply delaying instant decoupling. ${}^4$He however only increases by 68\% of that caused by delayed decoupling. This is due to ${}^4$He being governed by the $p\leftrightharpoons n$ reactions, which take place while $T_\nu/T_\gamma$ is still increasing. This supports the previous assumption, that the only major difference between the cosmological calculation of AlterBBN and this code, is their inclusion of incomplete neutrino decoupling.

\subsection{Reaction network}

Since the difference in background introduces minimal abundance corrections, the discrepancy between the AlterBBN results and mine must unsurprisingly be due to differences in the reaction network. To see where these differences are we can plot the relative difference between abundances as they evolve during BBN, displayed on figure \ref{fig:AlterBBNdeltaY}. 
\begin{figure}[ht]
    \includegraphics[width=5.1in]{figures/AlterBBNdeltaY.pdf}
    \caption{Relative deviation on abundances between this code and AlterBBN, during the various stages of BBN. To rule out cosmological differences, the AlterBBN background parameters are used for both calculations. Plot is linear in the region $\pm 10^{-3}$ and logarithmic outside}
    \label{fig:AlterBBNdeltaY}
\end{figure}


Starting at the high temperature limit, we see how the differences in early $p\leftrightharpoons n$ rates cause a deviation in proton and neutron abundance. This deviation corresponds exactly to that of the final ${}^4$He abundance. We can therefore conclude that the difference in $Y_p$ is purely due to differences in this rate. 

The next notable difference is the relative increase and subsequent drop in ${}^4\text{He}$ abundances, which takes place between 2 and 20 seconds. This matches the time period where in which the relative strength of the ${}^3\text{He}+n\rightarrow {}^4\text{He}$ rate is greatest compared to that of other reactions for ${}^4$He production. Suspecting this particular reaction is further supported by AlterBBN using a linear approximation for the rate, stemming from the original Wagoner code\cite{Wagoner69}. Though inaccurate, this doesn't actually have a significant impact on final abundances since the reaction only dominates ${}^4\text{He}$ production at very early times. 

Following this we see the most striking feature, which are the periodic spikes in relative abundance affecting almost every single nuclei. This jaggedness originates in AlterBBN, and is caused by their implementation of certain reaction rates. As described in section \ref{sec:pna}, I use smooth fits of experimental data to obtain appropriate values for the reaction rates at any temperature. With every one of these rates stemming from the most recent snapshot of the REACLIB database\cite{REACLIB} available in pynucastro\cite{pynucastro2}. AlterBBN on the other hand uses a manually curated list of reaction rates, with a myriad of different sources and implementations. A few of these rates are tabulated, and to obtain values between the tabulated temperature steps, AlterBBN simply uses nearest neighbor interpolation. This is hard coded as a series of consecutive else/if statements, which return the value of the reaction rate in discreet steps according to the nearest tabulated value. The spikes happen whenever these rates assume a new value. As an example AlterBBN uses a tabulated ${}^2\text{H}+p\rightarrow {}^3\text{He}$ rate from \textcite{Coc_et_al_2015}, which is responsible for the small early spikes in ${}^3\text{He}$, and other nuclei that depend on its abundance. This can be confirmed by noting that the early spikes peak at 1.875, 1.625, 1.375, and 1.125 GK, which happen to be the exact midpoints between the tabulated values which are at 2, 1.75, 1.50, 1.25, and 1 GK. 


These spikes obscure what processes responsible for the deviation in nuclear abundances, but for ${}^7\text{Li}$ and ${}^7\text{Be}$ in particular we can make very good guesses as to what these reactions are. The primary reactions responsible for creating and destroying ${}^7\text{Li}$ are ${}^3\text{H}+{}^4\text{He}\rightarrow {}^7\text{Li}$ and ${}^7\text{Li} + p\rightarrow 2{}^4\text{He}$, with ${}^7\text{Be}$ being governed by ${}^3\text{He}+{}^4\text{He}\rightarrow {}^7\text{Be}$ and ${}^7\text{Be} + n\rightarrow {}^7\text{Li} + p$. Compared to these, all other reactions are completely insignificant, which can be seen on \cref{fig:Be7destruct,fig:Be7create,fig:Li7destruct,fig:Li7create}. This makes these reactions the most like cause of the deviations, which is further supported by AlterBBN and my network using different rates for each one of these them.

\subsubsection{Modifying the Network with AlterBBN rates}

To test these claims we can replace the default rates in my network with the four ${}^7\text{Li}$ and ${}^7\text{Be}$ rates from AlterBBN as well as the previously mentioned ${}^3\text{He}+n\rightarrow {}^4\text{He}$ rate. Due to the modular nature of my reaction network this is quite easy to accomplish, with the resulting abundances being shown on \cref{tab:Alterratesabun}.
\begin{table}[ht]
    \begin{tabular}{l|llllll}
               & $Y_p \times 10$ & \hspace{-0.34em}$^{2}$H$ \times 10^{5}$ & \hspace{-0.34em}$^{3}$He$ \times 10^{5}$ & \hspace{-0.34em}$^{7}$Li $ \times 10^{10}$& \hspace{-0.34em}$^{6}$Li $ \times 10^{14}$& \hspace{-0.34em}$^{7}$Be $ \times 10^{10}$\\ \hline
    Default rates  & 2.473    & 2.540 & 1.041 & 4.699 & 0.754 & 4.398    \\ \hline
    \begin{tabular}[c]{@{}l@{}}Rates from  \\ AlterBBN \end{tabular} & 2.473    & 2.540 & 1.041 & 5.133 & 0.754 & 4.837    \\ \hline
    AlterBBN & 2.474            & 2.467 & 1.034 & 5.363 & 1.087 & 5.075   \\ %\hline
    $\quad +/-$ & 0.003           & 0.038 & 0.016 & 0.352 & 1.085 & 0.343     
    \end{tabular}
    \caption{Final abundances for the default network, the network using select AlterBBN rates, and the results from AlterBBN}
    \label{tab:Alterratesabun}
\end{table}
As expected only the abundances of ${}^7\text{Li}$ and ${}^7\text{Be}$ are affected, since the total abundances of light nuclei involved in the four reactions are much higher. 

Though the modified rates bring ${}^7\text{Li}$ and ${}^7\text{Be}$ abundances within the estimated uncertainty of AlterBBN, they still deviate significantly. To explain this figure \ref{fig:AlterBBNdeltaY} can be recreated using the modified reaction network, resulting in figure \ref{fig:AlterratesBBNdeltaY}. Looking at the early abundances we confirm that the initial difference in light nuclei abundance was indeed caused by the ${}^3\text{He}+n\rightarrow {}^4\text{He}$ rate. Additionally, we see how the spiking affects both ${}^7\text{Li}$ and ${}^7\text{Be}$ right up until the point at which they settle into their final abundances, making this the most likely explanation for the remaining discrepancy.

\begin{figure}[ht]
    \includegraphics[width=5.1in]{figures/AlterratesBBNdeltaY.pdf}
    \caption{Same as \cref{fig:AlterBBNdeltaY}, but with the network using the same rates as AlterBBN for the reactions ${}^3\text{He}+n\rightarrow {}^4\text{He}$, ${}^7\text{Li}$ are ${}^3\text{H}+{}^4\text{He}\rightarrow {}^7\text{Li}$, ${}^7\text{Li} + p\rightarrow 2{}^4\text{He}$, ${}^3\text{He}+{}^4\text{He}\rightarrow {}^7\text{Be}$, and ${}^7\text{Be} + n\rightarrow {}^7\text{Li} + p$}
    \label{fig:AlterratesBBNdeltaY}
\end{figure}



\section{Final abundances}

With differences between this and AlterBBN established, we can look at the correlation with the other codes, and more importantly the actual observations. 
\begin{table}[ht]
    \begin{tabular}{l|llllll}
               & $Y_p \times 10$ & \hspace{-0.34em}$^{2}$H$ \times 10^{5}$ & \hspace{-0.34em}$^{3}$He$ \times 10^{5}$ & \hspace{-0.34em}$^{7}$Li $ \times 10^{10}$\\ \hline
    Observations  & 2.453$\pm$0.034    & 2.527$\pm$0.030 & <1.1$\pm$0.2 & 1.6$\pm$0.3    \\ \hline
    This work  & 2.472            & 2.536 & 1.040 & 4.706    \\ \hline
    AlterBBN & 2.474$\pm$0.003 & 2.467$\pm$0.038 & 1.034$\pm$0.016 & 5.363$\pm$0.343    \\ \hline
    PRIMAT & 2.4709$\pm$0.0017   & 2.459$\pm$0.036 & 1.074$\pm$0.026 & 5.623$\pm$0.247   \\ \hline
    PArthENoPE& 2.4687$\pm$0.0012     & 2.51$\pm$0.06 %&   & 5.133      
    \end{tabular}
    \caption{Observed final abundances\cite{Yp_Aver_2021}\cite{deuterium_Cooke_2018}\cite{Allobsabun}, and the predictions by various BBN codes\cite{AlterBBN}\cite{PRIMAT}\cite{PArthENoPE},}
    \label{tab:Obsabun}
\end{table}
From \cref{tab:Obsabun}, we see how the abundances predicted by this code lie within the span of the values obtained by other codes. PRIMAT and PArthENoPE both have very precise estimates for $Y_p$ which are mutually consistent. This is due to them both calculating the exact $p\leftrightharpoons n$ rates with very high precision. This is omitted by AlterBBN and this code since it is very time-consuming, and not required due to the much high observational uncertainty. 

Deuterium stands out with, my implementation being outside the quoted uncertainty of both AlterBBN and PRIMAT. As always this is most likely due to a different choice of reaction rates. In particular the rate for ${}^2\text{H}+n\rightarrow {}^3\text{He}$, for which Reaclib uses a theoretical rate from \textcite{Reaclibdphe3_2004}, as opposed to the other codes which use more recent experimental rates. For AlterBBN and the 2018 snapshot of PRIMAT, this results in lower deuterium abundances. Yet it matches PArthENoPE since they use a recent rate from LUNA, which is actually closer the old theoretical rate than previous experimental rates\cite{Luna_impact_Pisanti_2021}. 

For $^{3}$He my results match those of AlterBBN, with PRIMAT being slightly higher. This might have been an issue if not for the fact that the observational constraint is extremely poor. 

${}^7\text{Li}$ is where we see the largest deviation. Relative to AlterBBN we know this is due to different reactions rates, and this is also the most probable explanation for PRIMAT. But every single code is nowhere near the observational value, which is of course the infamous cosmological lithium problem.


\subsubsection{$\eta$ and the Lithium problem}
The most well known illustration of the lithium problem is obtained by plotting the final abundances as a function of primary free parameter of BBN $\eta$. At any given time the balance of various reaction rates is given by the ratio of baryon density and temperature. Barring radically different physics any cosmological solution to the lithium problem will involve modifying this ratio, for which $\eta$ is a direct proxy. 



\begin{figure}[ht]
    \includegraphics[width=5.1in]{figures/etaplot.pdf}
    \caption{Resulting final abundances for different values of $\eta$. Gray lines represent the value for $\eta$ from the most recent Planck results\cite{Planck}. Orange represent the observational constraints given in \cref{tab:Obsabun}, and the remaining lines are the abundances produced by this code with uncertainties estimated based on comparable codes}
    \label{fig:etaplot}
\end{figure}
%\section{Nuclear solutions to the Lithium problem?}