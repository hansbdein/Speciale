%% ============================================================================
%%
%%  Master's thesis
%%
%%  Author: FORNAVN EFTERNAVN
%%
%%  Chapter 2: Another topic....
%% ============================================================================

\chapter{BBN code}
\label{chap:BBNcode}

\section{History of BBN codes}
\label{sec:BBN_history}

\subsection{Wagoner}
\label{sec:Wagoner}

\subsection{Kawano}
\label{sec:Kawano}

\subsection{Modern codes}
\label{sec:Modern_codes}
PArthENoPE, AlterBBN, PRIMAT


\subsection{AlterBBN}
\label{sec:AlterBBN}

AlterBBN is written in c and based on Kawano's NUC123. It maintains the same basic structure and integration method, though it uses natural units for everything but the reaction network. However they defines energy in GeV rather than MeV. What seperates AlterBBN for other codes is that as the name implies, it allows the use of alternate cosmological models and parametes. Therefore this code is especially well suited for testing the effects these alterrations have on final abundances.