%% ============================================================================
%%
%%  Master's thesis
%% 
%%  Author: FORNAVN EFTERNAVN
%%
%%  Chapter 1: Coffee
%% ============================================================================

\chapter{BBN physics and cosmology}
\label{chap:theory}

To understand the process of Big Bang nucleosynthesis, we must examine the intersection between Cosmology, thermodynamics, particle, and nuclear physics. Though this might seem daunting, it turns out that the unique conditions during this epoch allow for extensive simplifications of this otherwise monumental task.



% ~~~~~~~~~~~~~~~~~~~~~~~~~~~~~~~~~~~~~~~~~~~~~~~~~~~~~~~~~~~~~~~~~~~~~~~~~~~~~
% SECTION
% ~~~~~~~~~~~~~~~~~~~~~~~~~~~~~~~~~~~~~~~~~~~~~~~~~~~~~~~~~~~~~~~~~~~~~~~~~~~~~
\section{Determining background parameters}
\label{sec:Background}

\subsection{Temperature and scale factor}
\label{ssec:cosmology}

BBN takes place after inflation while the universe is still radiation dominated. This can be described by the Friedman equation, which can be further simplified with the reasonable approximation, that both curvature and the cosmological constant are zero. 

\begin{align}
    H^2=\left(\frac{\dot{a}}{a}\right)^2=\frac{8\pi G}{3}\rho_{tot}
\end{align}

With $\rho_{tot}$ referring to the total energy density of photons, leptons and baryons.
\begin{align}
    \rho_{tot}=\rho_{\gamma}+\rho_{\nu}+(\rho_{e^-}+\rho_{e^+})+\rho_{b}
\end{align}

To find an expression for the temperature evolution, we use energy conservation. We can consider the neutrinos as decoupled during BBN, and so the photon temperature will be determined by the remaining components. Since this point the universe is very much homogeneous and isotropic, we utilize the fluid equation for adiabatic expansion. 

\begin{align}
    \dot{\rho_{set}}+3\frac{\dot{a}}{a}(\rho_{set} + P_{set})=0
\end{align}

With $\rho_{set}$ being the density of none-decoupled components and $P_{set}$ being their pressures.


\begin{align}
    \rho_{set}=\rho_{\gamma}+(\rho_{e^-}+\rho_{e^+})+\rho_{b}
    \eqsep P_{set}=P_{\gamma}+(P_{e^-}+P_{e^+})+P_{b}
\end{align}

With this we can set up differential equations describing the time evolution of the scale factor and photon temperature. 


\begin{align}
    \diff{T}{t}=-3H\frac{\rho_{set}(T,a) + P_{set}(T,a)}{\diff{\rho_{set}(T,a)}{T}} \eqsep \diff{a}{t}=a\sqrt{\frac{8\pi G}{3}\rho_{tot}(T,a)}
    \label{eq:dBackground}
\end{align}

\subsection{Additional parameters}
Most BBN codes are based on the original code by Wagoner described in section \ref{sec:BBN_history}. These don't track the scale factor, but instead use the quantity $h$.

\begin{align}
    h=M_u\frac{n_{b}}{T^3_9}
\end{align}

$M_u$ being atomic mass units, $n_b$ the baryon number density, and  $T_9$ the temperature in $10^9$ Kelvin. This quantity was useful since it stays approximately constant throughout BBN, while being easy to directly convert to baryon density. However, with modern computers this numerical simplicity is inconsequential, and as such it is more reasonable to track the scale factor.
The electron chemical potential $\phi$ was also tracked by the Wagoner code and its successors. The only effect of this is ensuring a non-zero electron density after reheating. We can easily set this to 0, as the impact will be 3 orders of magnitude lower than the already miniscule impact of the baryon density.

\textcolor{orange}{Neutrino temperature?}


\section{Energy densities and pressure}


\subsection{Photons}


The energy density of photons as a function of temperature can be found by integrating over the energy distribution given by Planck's law. 


\begin{align}
    \rho_\gamma(T)=\int_{0}^{\infty} \frac{\omega^3}{\pi^2}\frac{1}{e^{\omega/T}-1}d\omega =  \frac{T^4}{\pi^2}\int_{0}^{\infty}\frac{u^3}{e^{u}-1}du
\end{align}

This integral is a well know representation of the Riemann Zeta function \cite[\href{https://dlmf.nist.gov/25.5.E1}{(25.5.1)}]{NIST:DLMF}.

\begin{align}
    \rho_\gamma(T)=\frac{T^4}{\pi^2}\Gamma(4)\zeta(4)=\frac{\pi^2}{15}T^4
    \label[equation]{rhogamma}
\end{align}

From this we can easily define the pressure and temperature derivative.

\begin{align}
    \diff{\rho_\gamma(T)}{T}=\frac{4}{15}\pi^2T^3 \eqsep P_\gamma(T)=\frac{\rho_\gamma(T)}{3}
\end{align}




\subsection{Neutrinos}




\subsection{Electrons and positrons}

\subsection{Baryons}

\lipsum




% ~~~~~~~~~~
% SUBSECTION
% ~~~~~~~~~~
\section{Nuclear reactions}
\label{sec:nucleartheory}

\subsection{Proton $\leftrightharpoons$ neutron rate}




Warning test\fxwarning{This is a warning!}

\lipsum

\section{Initial conditions}