








\chapter{Conclusion}
\label{chap:Conclusion}
%I have created a new BBN code based purely on fundamental cosmology, and standardized public reaction rates. 

We derived every equation necessary for the computation of background parameters with sufficient precision. This includes an updated version of the classic derivation of $e^{-}e^{+}$ density and pressure originally performed by Chandrasekhar \cite{Chandrasekhar}, and a correction to the initial time as presented by Wagoner\cite{Wagoner67}, as well as the incomplete correction by Sharpe\cite{sharpe2021big}. 

\noindent A BBN code was created to fulfill the five requirements stated in the introduction.
\begin{itemize}
    \item \textbf{Accessibility}: APODORA requires nothing except Python and publicly available packages. It is currently available on my personal GitHub page \url{github.com/hansbdein} with a clean and properly documented version being released in the near future at \url{github.com/AarhusCosmology}.
    \item \textbf{Accuracy}: APODORA was demonstrated to have numerical precision equal to or superior to that of contemporary codes. The accuracy of $Y_p$ is slightly lower than that of PRIMAT and \textsc{PArthENoPE} due to the lack of incomplete neutrino decoupling and a direct calculation of the $p\leftrightharpoons n$ rate. This is acceptable since the accuracy is still well within the observational constraints and that of other codes without an explicit focus on accurate $Y_p$ predictions.
    \item \textbf{Alacrity}: Using optimal precision parameters, APODORA is just as fast as other BBN codes with an average runtime of 0.61 seconds on an Intel i3-1115G4 @ 3.00GHz.
    \item \textbf{Agnostic reaction rates}: A reaction network was created using rates from the REACLIB database\cite{REACLIB}, eliminating any need for manual rate selection.
    \item \textbf{Adaptability}: APODORA was implemented in a modular fashion, with the reaction network and calculation of cosmological parameters being independent of the main integration routine. The main routine is in a Jupyter Notebook, allowing for easy access to every parameter at all times during the process of BBN. 
\end{itemize}




