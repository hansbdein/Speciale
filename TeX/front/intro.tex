%% ============================================================================
%%
%%  Speciale / Master's thesis
%%
%%  Author: FORNAVN EFTERNAVN
%%
%%  Introduction
%% ============================================================================

\chapter{Introduction}
\label{chap:intro}
This thesis is about nucleosynthesis

\section{A brief history of early nucleosynthesis}
A second after the Big Bang, the universe is at $10^10$K. Though quite hot, this temperature is low enough that neutrons and protons can no longer maintain thermodynamics equilibrium, freezing their ratio at one to five. At approximately 200 seconds the universe has cooled sufficiently so that high energy photons can no longer destroy deuterium. This causes almost all neutrons to be use in the creation of deuterium, which is rapidly converted to helium. Due to the delay caused by deuterium, the temperature and density of the universe will be to low to create any more than trace amounts of heavier elements. And so a few minutes after it began, primordial nucleosynthesis ends. Barring radioactive decay, the abundance of the various elements will remain unchanged, until the first star appear $10^8$ years later. 


